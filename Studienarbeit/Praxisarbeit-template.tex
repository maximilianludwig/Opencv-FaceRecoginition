\documentclass[12pt, a4paper]{scrbook}
\usepackage[utf8]{inputenc}
\usepackage{csquotes}
\usepackage[german]{babel}
\usepackage{hyperref}
\usepackage[onehalfspacing]{setspace}
\usepackage{geometry}
\usepackage{color}
\usepackage{listings}
\usepackage{graphicx}
\usepackage{acronym}
\usepackage[backend=biber,
bibstyle=numeric, 
citestyle=authortitle,
natbib=true, 
hyperref=true,
]{biblatex}
\addbibresource{lit.bib} 
\setcounter{secnumdepth}{4}
\setcounter{tocdepth}{4}
\geometry{
left=2.5cm,
right=2.5cm,
top=2.5cm,
bottom=2.5cm,
bindingoffset=5mm,
}
\pagestyle{empty}
\begin{document}
\include{Titelseite}
\setlength{\parindent}{0em} 
\renewcommand\thechapter{\Roman{chapter}}
\pagenumbering{Roman}
\let\cleardoublepage\relax
\section*{Erklärung}
Ich versichere hiermit, dass ich meine Projektarbeit mit dem Thema: ''Titel der großen Studienarbeit'' selbstständig verfasst und keine anderen als die angegeben Quellen und Hilfsmittel benutzt habe. 
\newline
\newline
\newline
\newline
---------------------------------------------       ------------------------------------------ \newline
Ort	\hspace{2cm}		Datum\hspace{3,5 cm}				    Unterschrift
\newpage
\section*{Abstract}


\newpage
\begingroup
\renewcommand*{\chapterpagestyle}{empty}
\pagestyle{empty}
\tableofcontents
\listoffigures
\section*{Abkürzungen}
\begin{acronym}[Bash]
 \acro{VCS}{Version Control System}
 \acro{CI/CD}{Continuous Integration/Continuous Deployment}
 \acro{API}{Application Programming Interface}
 \acro{DSL}{Domain Specific Language}
 \acro{KPI}{Key Performance Indicator}
 \acro{CRI}{Container Runtime Interface}
 \acro{JCasC}{Jenkins Configuration as Code}
 \acro{PV}{Persistent volume}
 \acro{PVC}{Persistent volume claim}
 \acro{NFS}{Network File System}
 \acro{iSCSI}{internal Small Computer System Interface}
\end{acronym}
\endgroup
\newpage
\pagestyle{plain}
\setcounter{page}{1}
\chapter{Einleitung}
''Several researchers have stated that facial expression recognition appears to play one of the most important roles in human communication'' 
\footcite[Vgl.][1]{FaceRec}
Dieses Zitat von Katherine B. Leeland gibt einen Einblick in die Relevanz der Emotionserkennung für den Menschen. Dabei ist diese Relevanz nicht erst in jüngerer zeit entstanden. Bereits Darwin stellte die Frage, ob von den Gesichtsausdrücken einer Person nicht auch der Emotionale Zustand abgeleitet werden kann.
\footcite[Vgl.][2]{FaceRec}

S 249 im anderen Buch steh merh technisches

% mehr zum allgemeinen von Emotionserkennung bevor Brückenschlag 
Nun ist diese Studienarbeit kein wissenschaftliches Werk über die psychologischen Emotionen die hinter verschiedenen Gesichtsausdrücken stehen, jedoch thematisiert diese Arbeit dieses Themengebiet indirekt. Es geht hier darum eine künstliche Intelligenz zu entwicklen, die erkennen kann in wie weit ein Pokerface einer Person anhand dessen Gesichtsausdruck vorliegt. Diese Aufgabe scheint nun erst ein Mal nicht viel mit dem oben bereits genannten Themengebiet zu haben, dem ist jedoch nicht so. Ein Pokerface wird im Allgemeinen als eine emotionsloser Gesichtsausdruck definiert. Dies impliziert, dass eine Person mit Pokerface keine Emotion zu erkennen gibt. Daher könnte bei einer nicht messbaren Emotion ein Pokerface vorliegen. Dies mittels KI zu testen und einen sogenannten ''Pokerface Detektor'' zu entwicklen ist daher Ziel dieser Arbeit. Mit diesem Pokerface Detektor sind verschiedenste Einsatzmögölichkeiten denkbar, im Folgenden einige mögliche Szenarien:
%Beispiele ausführen
\begin{itemize}
	\item{Polizeiverhöre}
\end{itemize}
Es ist denkbar, dass ein Detektor wie der, der in dieser Arbeit entwickelt werden wird bei Polizeiverhöhren eingesetzt wird. Für die Beamten kann es nicht immer direkt ersichtlich sein, ob der verhöhrte mit einem Pokerface lügt, oder doch die Wahrheit sagt. Da die Möglichkeiten dies zu prüfen ebenfalls nicht sehr zahlreich sind, wäre es eine Vereinfachung für die Beamten wenn eine einfache Webcam zusammen mit einem Computer reichen würde um die Lügner zu entlarven.
\begin{itemize}
	\item{Gerichtsverhandlungen}
\end{itemize}
Das zweite Einsatzgebiet ist ähnlich zu dem ersten. Bei Gerichtsverhandlungen gelten die gleichen Vorraussetzungen wie bei einem Verhöh der Polizei. Zwar müssen die hier vorgeladenen eine eidestattliche Erklärung abgeben nur die Wahrheit zu sagen, jedoch ist zu bezweifeln ob dies auch jeder so handhabt. 
Nun soll nicht der Eindruck entstehen dass das hier gebaute Werkzeug ein Lügendetektor ist. Es ist ebenfalls nicht möglich, dass von einem Pokerface immer auf eine Lüge geschlossen werden kann. Jedoch ist ein Pokerface ein Zeichen dafür, dass sich diese Person ihren emotionalen Zustand nicht anmerken lassen möchte. Und dies wiederum deutet eher daraufhin dass die Person nicht die Wahrheit sagt oder nur teilweise. 
Abgesehen von Einsatzgebieten die zur Entlarvung von Lügen führen kann auch ein klassicheres Szenario verwendet werden:
\begin{itemize}
	\item{Pokerspiel}
\end{itemize}
Es ist anzunehmen, dass der erste Begriff der mit dem Wort Pokerface in Verbindung gebracht wird, das Pokerspiel selber ist. Und auch in diesem kann ein Pokerface Detektor nützlich sein. So kann ein Mitspieler zum Beispiel mittels einer Kamera das Gesicht des Gegenübers scannen und analysieren ob ein Pokerface vorliegt oder nicht, und dementsprechend agieren.
Nachdem verschiedene Anwendungszenarien beleuchtet wurden wird im Folgenden die konkrete Forschungsfrage beleuchtet.
%Irgendwas vergessen?
\section{Aufgabenstellung}
In diesem Abschnitt soll nun noch die konkrete Forschungsfage behandelt werden. Das Projekt selber wird an der DHBW in Mannheim durchgeführt und betreut von Prof. Dr. Erckhard Kruse.
\newline
 Wie eingangs erwähnt soll mittels künstlicher Intelligenz ein Pokerface erkannt werden. Dafür wird wiederum eine Bilderkennungssoftware notwendig, die ebennfalls angefertigt werden soll. Dieses Softwareprodukt soll Bilder%und Videos?
die mit einer Webcam aufgenommen werden nach ihren Emotionen analysieren. Je nachdem welche emotionen gezeigt werden wird dann von der Software ein Rückschluss gezogen auf ein eventuell vorhandens Pokerface. Ein konkretes Einsatzgfebiert nach Abschluss der Entwickliung ist nicht vorgesehen, da es sich um ein Forschungsprojekt handelt. jedoch sind wie bereits beschrieben einige verschiedene Einsatzmöglochkeiten denkbar, an die das Werkzeug leicht angepasst werden kann. 

%\section{DevOps}
%\begin{figure}
%\includegraphics[width=\linewidth]{Bilder/DevOps-LifeCycle-Features-Naresh-IT.png}
%\caption{ DevOps Lifecyle \newline Quelle: https://nareshit.com/wp-content/uploads/2018/09/DevOps-LifeCycle-Features-Naresh-IT.png }
%\label{fig:Devops-Lifecycle}
%\end{figure}


\let\cleardoublepage\relax
\chapter{Theorie}
Gliederung Theorie
\begin{itemize}
\item unterschied es gibt face regocnition emotion regocnition
\item use cases dafür 
\item unser use case wie er reinpasst
\item emotion rec =  abspaltung der face regocnition 
\item   
\end{itemize}


%https://books.google.de/books?id=JxSOjPzd5IgC&printsec=frontcover&dq=emotion+recognition&hl=de&sa=X&ved=0ahUKEwj27vTp7pTnAhUjuqQKHeSxDZ4Q6AEIPzAC#v=onepage&q=emotion%20recognition&f=false

%Seite 3 ganz viele Emotionsdefis

\let\cleardoublepage\relax
\newpage
\chapter{Methode}
Diese Arbeit soll methodisch mit der MoSCoW Priorisierung bearbeitet werden. Diese Art der Priorisierung teilt die zu bearbeitenden Anforderungen in vier Kategorien ein:
\footcite[vgl.][90]{Projektmanagement}
\begin{itemize}
\item Must - Core Anforderungen die unbedingt umgesetzt werden müssen
\item Should - Anforderungen die ebenfalls umgesetzt werden müssen, jedoch um Nachhinein noch durch Change Request verändert werden können.
\item Could - Anforderungen die Nach den Must und Should Anforderungen umgesetzt werden sollen, sofern noch Ressourcen und Zeit vorhanden sind um diese zu bearbeiten
\item Won't - Anforderungen die nicht in diesem Projekt bzw. Release erfolgen, jedoch in einer zukünftigen Version bearbeitet werden sollen. 
\end{itemize}
\section{Vorgehen im ersten Teil}
Die erste Aufgabe, die automatisierte Erstellung einer CI/CD Pipeline, soll nun die folgenden funktionalen, so wie nicht-funktionalen MoSCoW Anforderungen erfüllen:
\begin{itemize}
\item Must
\begin{itemize}

\end{itemize}
\item Should
\begin{itemize}

\end{itemize}
\item Could
\begin{itemize}

\end{itemize}
\item Won't
\begin{itemize}
\end{itemize}
\end{itemize}
Dabei sollen die einzelnen Anforderungen entsprechend ihrer Priorität abgearbeitet werden. So kann am Ende der Erfolg der Arbeit deutlich besser eingeordnet werden
\section{Vorgehen im zweiten Teil}
Bezüglich des zweiten Teils der Arbeit, dem automatisierten Deployment der Applikation, sollen folgende Anforderungen mit Ihrer Priorisierung aufgestellt werden:

\let\cleardoublepage\relax
\chapter{Ergebnis}
\section{test1}
\subsection{test2}
\subsubsection{test3}
\paragraph*{Gitea Setup}

\let\cleardoublepage\relax
\chapter{Diskussion}
Das nunmehr letzte Kapitel soll sich mit der kurzen Zusammenfassung der Ergebnisse des letztens Teils und deren Bewertung widmen. Des Weiteren sollen die angewandten Methoden reflektiert werden, offene Fragen beantwortet und auch weitere Punkte aufgezeigt werden die verbessert oder noch implementiert werden können. Dazu soll zunächst die Ergebnisse kurz zusammengefasst werden.

\let\cleardoublepage\relax
\pagestyle{empty}
\newpage
\pagestyle{empty}
\printbibheading
\printbibliography[type=book,heading=subbibliography,title={Literaturquellen}]
\printbibliography[type=misc,heading=subbibliography,title={Online Quellen}]
\pagestyle{empty}
\newpage
\pagestyle{empty}

\end{document}
